% Metódy inžinierskej práce

\documentclass[10pt,twoside,slovak,a4paper]{article}

\usepackage[slovak]{babel}
%\usepackage[T1]{fontenc}
\usepackage[IL2]{fontenc} % lepšia sadzba písmena Ľ než v T1
\usepackage[utf8]{inputenc}
\usepackage{graphicx}
\usepackage{url} % príkaz \url na formátovanie URL
\usepackage{hyperref} % odkazy v texte budú aktívne (pri niektorých triedach dokumentov spôsobuje posun textu)

\usepackage{cite}
%\usepackage{times}

\pagestyle{headings}

\title{Postavy a charaktery v hrách poháňané umelou inteligenciou\thanks{Semestrálny projekt v predmete Metódy inžinierskej práce, ak. rok 2021/22, vedenie: Vladimír Mlynarovič}} % meno a priezvisko vyučujúceho na cvičeniach

\author{Miroslava Mäsiariková\\[2pt]
	{\small Slovenská technická univerzita v Bratislave}\\
	{\small Fakulta informatiky a informačných technológií}\\
	{\small \texttt{xmasiarikova@stuba.sk}}
	}

\date{\small x. november 2021} % upravte



\begin{document}

\maketitle

\begin{abstract}

V súčasnosti je umelá inteligencia jeden z hlavných nástrojov na zlepšenie hráčskeho zážitku v hrách. Umelá inteligencia sa v hrách zameriava predovšetkým na tri základné sekcie: schopnosť pohybovať postavami, schopnosť rozhodovať kde a ako sa pohybovať a schopnosť myslieť strategicky. V mojom článku sa zameriam konkrétne na postavy neovládané hráčom, ale umelou inteligenciou. Úlohou týchto postáv je spraviť hru pre hráča ťažšou a zaujímavejšou. Charakter týchto postáv je v hrách rôzny, niektoré majú za úlohu hru iba oživiť a nerobia žiadne špeciálne úkony, zatiaľ čo iné sa samy rozhodujú, pohybujú a skúmajú prostredie okolo seba. Takéto postavy sú schopné učiť sa od ostatných hráčov v hre. V mojom článku viac priblížim význam týchto postáv v hrách, rozoberiem aké algoritmy umelej inteligencie sú potrebné pri modelovaní, správaní a rozhodovaní týchto postáv, s akými problémami sa môžeme stretnúť a ako sa takéto postavy môžu ďalej v hre vyvíjať. Taktiež sa v stručnosti zameriam aj na vývoj umelej inteligencie od prvej hry využívanej umelú inteligenciou až po súčasnosť. 

\end{abstract}


\section{Čo je to NPC?}  \label{nejaka}

\begin{figure*}[tbh]
\centering
%\includegraphics[scale=1.0]{diagram.pdf}

\end{figure*}


\section{Načo sa NPC v hrách využívajú?} \label{ina}


\section{Prečo sú NPC v hrách dôležité?} \label{dolezita}


\section{Algoritmy umelej inteligencie, ktoré sa využívajú pri modelovaní, správaní a rozhodovaní NPC} \label{dolezitejsia}

\section{Problémy s ktorými sa môžeme stretnúť pri používaní umelej inteligencie v hrách} \label{dolezita}

\section{Ako sa NPC v hrách vyvíjajú} \label{dolezita}

\section{Vývoj umelej inteligencie od začiatku až po súčastnosť} \label{dolezita}


\section{Záver} \label{zaver} % prípadne iný variant názvu


\footnote{Niekedy môžete potrebovať aj poznámku pod čiarou.}


Môže sa zdať, že problém vlastne nejestvuje\cite{Coplien:MPD}, ale bolo dokázané, že to tak nie je~\cite{Czarnecki:Staged, Czarnecki:Progress}. Napriek tomu, aj dnes na webe narazíme na všelijaké pochybné názory\cite{PLP-Framework}. Dôležité veci možno \emph{zdôrazniť kurzívou}.



%\acknowledgement{Ak niekomu chcete poďakovať\ldots}


% týmto sa generuje zoznam literatúry z obsahu súboru literatura.bib podľa toho, na čo sa v článku odkazujete
\bibliography{literatura}
\bibliographystyle{plain} % prípadne alpha, abbrv alebo hociktorý iný
\end{document}
