% Metódy inžinierskej práce

\documentclass[10pt,twoside,slovak,a4paper]{article}

\usepackage[slovak]{babel}
%\usepackage[T1]{fontenc}
\usepackage[IL2]{fontenc} % lepšia sadzba písmena Ľ než v T1
\usepackage[utf8]{inputenc}
\usepackage{graphicx}
\usepackage{url} % príkaz \url na formátovanie URL
\usepackage{hyperref} % odkazy v texte budú aktívne (pri niektorých triedach dokumentov spôsobuje posun textu)

\usepackage{cite}
%\usepackage{times}

\pagestyle{headings}

\title{Postavy a charaktery v hrách poháňané umelou inteligenciou\thanks{Semestrálny projekt v predmete Metódy inžinierskej práce, ak. rok 2021/22, vedenie: Vladimír Mlynarovič}} % meno a priezvisko vyučujúceho na cvičeniach

\author{Miroslava Mäsiariková\\[2pt]
	{\small Slovenská technická univerzita v Bratislave}\\
	{\small Fakulta informatiky a informačných technológií}\\
	{\small \texttt{xmasiarikova@stuba.sk}}
	}

\date{\small x. november 2021} % upravte

%\includegraphics[scale=1.0]{diagram.pdf}

\begin{document}

\maketitle

\begin{abstract}

V súčasnosti je umelá inteligencia jeden z hlavných nástrojov na zlepšenie hráčskeho zážitku v hrách. Umelá inteligencia sa v hrách zameriava predovšetkým na tri základné sekcie: schopnosť pohybovať postavami, schopnosť rozhodovať kde a ako sa pohybovať a schopnosť myslieť strategicky. V mojom článku sa zameriam konkrétne na postavy neovládané hráčom, ale umelou inteligenciou. Úlohou týchto postáv je spraviť hru pre hráča ťažšou a zaujímavejšou. Charakter týchto postáv je v hrách rôzny, niektoré majú za úlohu hru iba oživiť a nerobia žiadne špeciálne úkony, zatiaľ čo iné sa samy rozhodujú, pohybujú a skúmajú prostredie okolo seba. Takéto postavy sú schopné učiť sa od ostatných hráčov v hre. V mojom článku viac priblížim význam týchto postáv v hrách, rozoberiem aké algoritmy umelej inteligencie sú potrebné pri modelovaní, správaní a rozhodovaní týchto postáv, s akými problémami sa môžeme stretnúť a ako sa takéto postavy môžu ďalej v hre vyvíjať. Taktiež sa v stručnosti zameriam aj na vývoj umelej inteligencie od prvej hry využívanej umelú inteligenciou až po súčasnosť. 

\end{abstract}


\section{Čo je to NPC?}  \label{nejaka}
NPC je anglická skratka pre non-player character. NPC sú postavy v hrách, ktoré nie sú ovládané hráčom, ale fungujú na základe umelej inteligencie, tzn. ovláda ich sám počítač. Tieto postavy sú nevyhnutnou súčasťou na zatraktívnenie akejkoľvek hry. Ich hlavnou úlohou je spraviť hru pre hráča atraktívnou, zaujímavou a komplikovanejšou, pretože jednoduchá hra pre hráča nie je zaujímavá. Charakter týchto postáv by sa dal rozdeliť na dve skupiny. V prvej skupine sa nachádzajú postavy, ktorých úlohou je hru iba oživiť a nerobia žiadne špeciálne úlohy. Takéto postavy iba dopĺňajú prostredie v ktorom sa väčšinou pohybujú. V druhej skupine sa nachádzajú postavy, ktoré sú schopné ďalej sa učiť od ostatných hráčov v hre. Tieto postavy sú schopné samy sa pohybovať, rozhodovať a strategicky myslieť. Ich úlohou je sťažiť reálnym hráčom hru a nedovoliť im len tak jednoducho vyhrať. Jedinou nevýhodou týchto postáv je, že nie sú ešte dostatočne inteligentné. Väčšina týchto postáv v hrách vedie iba jednoduché dialógy alebo ovplyvňuje priebeh deja na základe doterajších skúseností s hráčmi. Ako jednoduchý príklad uvediem napríklad jednoduchú kartovú hru. V takýchto hrách sa okrem hráčov musí nachádzať aj nejaká neutrálna osoba, resp. rozhodca, ktorá bude dozerať na pravidlá hry. Práve touto neutrálnou osobou je postava, ktorá nie je ovládaná žiadnym hráčom, ale počítačom. Tento rozhodca vedie jednoduché dejové dialógy s reálnymi hráčmi, kontroluje dodržiavanie pravidiel a dotvára celkovú atmosféru hry. 
\begin{figure*}[tbh]
\centering
\end{figure*}
\section{Typy NPC} \label
V V hernom svete sa môže objaviť niekoľko typov týchto nehrateľných postáv. A to funkčné postavy, opačná kocka, spoluhráči a lídri. 
\begin{itemize}
\item funkčné postavy 
	\begin{itemize}
	\item Tieto postavy sú najjednoduchšie z vyššie vymenovaných. Nepotrebujú veľkú autonómiu ani inteligenciu, pretože iba odpovedajú na príslušné otázky. Avšak pokročilejšie postavy z tejto skupiny môžu medzi sebou aj komunikovať nejakým jazykom alebo zadávať či vykonávať rôzne úlohy, aby dosiahli v hre realistickejší efekt. Vďaka umelej inteligencií, ktorú využívajú sa môžu v hre ďalej učiť prostredníctvom hlasu. Najjednoduchším príkladom je, že ak poviete tejto postave, že jablká sú červené a banány sú žlté, tak si to táto postava zapamätá a v jej vedomostiach budú jablká červená a banány žlté. Ak jej však veľa ľudí povie, že majú aj zelené jablká, potom túto informáciu spracuje a pridá ju k svojim vedomostiam. Toto bol avšak iba jednoduchý príklad pre pochopenie danej problematiky. V hre je učenie tejto funkčnej postavy skôr o znalostiach potrebných v tomto hráčskom svete, ako napríklad kde kúpiť zbraň, či sa ju oplatí kúpiť, ktorý úkryt je pre ňu vhodný a podobne. 
	\end{itemize}
\item opačná kocka
	\begin{itemize}
	\item V hrách sa veľmi často stretneme s nejakým šéfom. Tieto postavy sú vždy riadené práve počítačom. Týchto šéfov môžeme rozdeliť na mini šéfa, super šéfa a finálneho šéfa. Mini šéf nevyužíva veľmi zložitú technológiu umelej inteligencie. Ide iba o obyčajného nepriateľa na základnej úrovni, ktorý sa musí dostať k reálnemu hráčovi dostatočne blízko, v istom časovom intervale a spôsobiť mu nejakú ujmu, aby mu hru skomplikoval. Super šéf je voliteľný nepriateľ pre sťaženie hry a finálny šéf je nepriateľ na konci hry. Super šéf aj finálny šéf sú pre hráča ťažkí nepriatelia, ktorí sa objavia v momente, keď sa môže zdať, že hráč hru vyhrá. Pre tieto postavy je už potrebná pokročilejšia úroveň umelej inteligencie. Pri modelovaní, resp. vývoji, sa hry musia často testovať. Vtedy sa zavolá istá skupina dobrovoľníkov a ich úlohou je vyškoliť týchto nepriateľov, resp. šéfov v hre. Pretože tento nepriateľ sa učí predovšetkým na vlastných chybách a slabinách druhej strany, teda na strane hráča. Vďaka takémuto testovaniu sa nepriateľ v hre dokáže vyvinúť do takej fázy, aby sa stal čo najmenej poraziteľným. 
	\end{itemize}
\item spoluhráči
	\begin{itemize}
	\item Postava spoluhráča, ktorú ovláda umelá inteligencia je veľmi ťažko kontrolovateľná, pretože tieto postavy dbajú na spoluprácu medzi sebou. A ak sa nachádza v hre veľa chýb alebo nezrovnalostí, tak tieto postavy nefungujú správne a hra môže dokonca zlyhať. Preto je ťažké nechať umelú inteligenciu vykonávať postavu spoluhráča. V súčasnosti sa tieto postavy ani veľmi v hráčskom prostredí nevyužívajú. Keďže umelá inteligencia funguje aj na hĺbkovom učení je vysoko pravdepodobné, že v budúcnosti budú tieto postavy zlepšovať svoju silu a spoluprácu medzi sebou vďaka ich autonómnemu tréningu. Kedy sa budú postavy učiť bez dozoru a na sebe samých. 
	\end{itemize}
\item lídri
	\begin{itemize}
	\item Posledným typom neovládateľných postáv sú lídri. Tieto postavy by sa dali prirovnať k súčasnému programu inteligentného robota. Majú svoje samostatné myslenie a učenie. Lídri sa pozerajú na celkovú situáciu v hre, analyzujú zmeny a robia rôzne úsudky. Napríklad v hre s názvom Civilizácia, súťažia hráči s takouto postavou a bojujú o územie. Líder sa tu sám rozhoduje na základe aktuálnej situácie, napríklad nadväzuje diplomatické vzťahy, vedie vojny, posiela vojakov do boja, ochraňuje domovy a podobne. Táto časť umelej inteligencie avšak nie je ešte dokonalá a na konci hry sa môže stať, že postava sa bude rozhodovať už iba náhodným výberom. 
	\end{itemize}
\end{itemize}

\section{Záver} \label{zaver} % prípadne iný variant názvu


%\footnote{Niekedy môžete potrebovať aj poznámku pod čiarou.}



%\acknowledgement{Ak niekomu chcete poďakovať\ldots}


% týmto sa generuje zoznam literatúry z obsahu súboru literatura.bib podľa toho, na čo sa v článku odkazujete
%\bibliography{literatura}
\bibliographystyle{plain} % prípadne alpha, abbrv alebo hociktorý iný
\end{document}
